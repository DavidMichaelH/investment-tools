\documentclass[11pt]{article}
\usepackage{amstext,amsmath,amssymb,amsthm}
\usepackage{latexsym}
\usepackage{exscale}
\usepackage{color}



\usepackage[margin=1in]{geometry} 
\usepackage{amsmath,amsthm,amssymb, graphicx, multicol, array}

%Plot stuff
\usepackage{pgfplots}
\pgfplotsset{compat=1.6}

\pgfplotsset{soldot/.style={color=blue,only marks,mark=*}} \pgfplotsset{holdot/.style={color=blue,fill=white,only marks,mark=*}}



\RequirePackage[cp1251]{inputenc}
%\RequirePackage[english,russian]{babel}
\RequirePackage{srcltx}

\numberwithin{equation}{section}
\sloppy
%\def\baselinestretch{1.2}Boch


\RequirePackage{srcltx}

\RequirePackage[cp1251]{inputenc}
%\RequirePackage[english,russian]{babel}


\numberwithin{equation}{section}

%\def\baselinestretch{1.2}Boch


\renewcommand\rq[1]{\textup{(\ref{#1})}}

\newcommand\ov{\overline}
\renewcommand\t{\widetilde}
\newcommand\h{\widehat}
\renewcommand\r{\rangle}
\renewcommand\l{\langle}

\newcommand\dsize{\displaystyle}
\newcommand\supp{\operatorname{supp}}
\newcommand\dist{\operatorname{dist}}
\newcommand\Ann{\operatorname{Ann}}
\renewcommand\Re{\operatorname{Re}}
\renewcommand\Im{\operatorname{Im}}
\renewcommand\div{\operatorname{div}}
%\newcommand\cal{\mathcal}

\newcommand\R{\mathbb{R}}
\newcommand \C{\mathbb{C}}
\newcommand\Q{\mathbb{Q}}
\newcommand\Z{\mathbb{Z}}
\newcommand\N{\mathbb{N}}
\renewcommand\S{\mathbb{S}}
\newcommand\A{{\bf A}}
\newcommand\M{{  M}}
\newcommand\tr{{\mbox tr}}

\newcommand\B{\mathcal{B}}
\newcommand\s{\mathcal{S}}

\newcommand\E{\mathbb{E}}
\renewcommand\P{\mathbb{P}}

\usepackage{mathtools}
\DeclarePairedDelimiter\ceil{\lceil}{\rceil}
\DeclarePairedDelimiter\floor{\lfloor}{\rfloor}


\newcommand\ddfrac[2]{\frac{\displaystyle #1}{\displaystyle #2}}

\newtheorem{Thm}{Theorem}[section]
\newtheorem{Lemma}[Thm]{Lemma}
\newtheorem{Cor}[Thm]{Corollary}
\newtheorem{Prop}[Thm]{Proposition}
\newtheorem{Prob}{Problem}[section]
\theoremstyle{remark}
\newtheorem{Soln}{Solution}[section]
\newtheorem{Rem}{Remark}[section]
\newtheorem{Ques}{Question}[section]
\newtheorem{Def}{Definition}[section]
\newtheorem{Conj}{Conjecture}[section]

\newtheorem{MyCom}{My Comment}[section]
\renewcommand\hat{\widehat}
\sloppy

\def\Xint#1{\mathchoice
	{\XXint\displaystyle\textstyle{#1}}%
	{\XXint\textstyle\scriptstyle{#1}}%
	{\XXint\scriptstyle\scriptscriptstyle{#1}}%
	{\XXint\scriptscriptstyle\scriptscriptstyle{#1}}%
	\!\int}
\def\XXint#1#2#3{{\setbox0=\hbox{$#1{#2#3}{\int}$ }
		\vcenter{\hbox{$#2#3$ }}\kern-.6\wd0}}
\def\ddashint{\Xint=}
\def\dashint{\Xint-}


\begin{document}
	
	
	
	
	
	
	
	\title{Notes}
	% \author{David Harper}
	\date{}
	% \email{dharper40@gatech.edu}
	%
	%
	%	\subjclass[2017]{}
	
	
	\maketitle 
	
	
	Unit vector of direction to manifold. 
	
	\[ X_{t+1} = A X_t + BN_{t}\]
	 
		\[ \E X_{t+1}X_{t+1}^T  = \E( A X_t + BN_{t} ) ( A X_t + BN_{t} )^T \]
 \[  =  A \E X_tX_t^T A^t  + B\E N_{t}N_{t}^T   \]
 where we use $\E N_t = 0$ and the independence of $X_t$ and $N_t$. Then $N_{t}N_{t}^T  = I$ gives 
	$$ \Sigma_{t+1} = A \Sigma_t A^T + B $$ 

 
		\[\Sigma= AXX^T + B N X  \]
	
	
	The plane of attraction is spanned by the eigenvectors corresponding to non-negative eigenvalues of $A$. 
	
	So look given $X_{t}$ the investment balance is given by
	 $$v = (P-I) X_{t}$$ be the vector which points
	
	Can we define an investment functional?  
	
	Let $W$ be a window value. We define input matrices 
	
	$$ [X_{t-W+1}| \hdots | X_{t} ] $$
	which we flatten and feed into an RNN or LSTM with the output $X_{t+1}$.  
	
	Given the correlation between the stock price returns we will first want to perform a dimension reduction.
	
	
	Next we will not predict prices but instead returns. We define the return at time $t$ for the $i$-th asset as
	$$ r_i(t) = \dfrac{x_{i}(t)-x_{i}(t-1)}{x_i(t-1)} $$
	and we define the return vector $R_t$ as the column vector with entries $r_{t}^i$. We first produce a dimension reduction on $R_t$. We can try out a variety of techniques from PCA to autoencoder. 
	
	
	
	
\end{document}